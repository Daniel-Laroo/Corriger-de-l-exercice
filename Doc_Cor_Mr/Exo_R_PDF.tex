% Options for packages loaded elsewhere
\PassOptionsToPackage{unicode}{hyperref}
\PassOptionsToPackage{hyphens}{url}
%
\documentclass[
]{article}
\usepackage{amsmath,amssymb}
\usepackage{lmodern}
\usepackage{iftex}
\ifPDFTeX
  \usepackage[T1]{fontenc}
  \usepackage[utf8]{inputenc}
  \usepackage{textcomp} % provide euro and other symbols
\else % if luatex or xetex
  \usepackage{unicode-math}
  \defaultfontfeatures{Scale=MatchLowercase}
  \defaultfontfeatures[\rmfamily]{Ligatures=TeX,Scale=1}
\fi
% Use upquote if available, for straight quotes in verbatim environments
\IfFileExists{upquote.sty}{\usepackage{upquote}}{}
\IfFileExists{microtype.sty}{% use microtype if available
  \usepackage[]{microtype}
  \UseMicrotypeSet[protrusion]{basicmath} % disable protrusion for tt fonts
}{}
\makeatletter
\@ifundefined{KOMAClassName}{% if non-KOMA class
  \IfFileExists{parskip.sty}{%
    \usepackage{parskip}
  }{% else
    \setlength{\parindent}{0pt}
    \setlength{\parskip}{6pt plus 2pt minus 1pt}}
}{% if KOMA class
  \KOMAoptions{parskip=half}}
\makeatother
\usepackage{xcolor}
\IfFileExists{xurl.sty}{\usepackage{xurl}}{} % add URL line breaks if available
\IfFileExists{bookmark.sty}{\usepackage{bookmark}}{\usepackage{hyperref}}
\hypersetup{
  pdftitle={Exercice\_R},
  pdfauthor={Dan},
  hidelinks,
  pdfcreator={LaTeX via pandoc}}
\urlstyle{same} % disable monospaced font for URLs
\usepackage[margin=1in]{geometry}
\usepackage{color}
\usepackage{fancyvrb}
\newcommand{\VerbBar}{|}
\newcommand{\VERB}{\Verb[commandchars=\\\{\}]}
\DefineVerbatimEnvironment{Highlighting}{Verbatim}{commandchars=\\\{\}}
% Add ',fontsize=\small' for more characters per line
\usepackage{framed}
\definecolor{shadecolor}{RGB}{248,248,248}
\newenvironment{Shaded}{\begin{snugshade}}{\end{snugshade}}
\newcommand{\AlertTok}[1]{\textcolor[rgb]{0.94,0.16,0.16}{#1}}
\newcommand{\AnnotationTok}[1]{\textcolor[rgb]{0.56,0.35,0.01}{\textbf{\textit{#1}}}}
\newcommand{\AttributeTok}[1]{\textcolor[rgb]{0.77,0.63,0.00}{#1}}
\newcommand{\BaseNTok}[1]{\textcolor[rgb]{0.00,0.00,0.81}{#1}}
\newcommand{\BuiltInTok}[1]{#1}
\newcommand{\CharTok}[1]{\textcolor[rgb]{0.31,0.60,0.02}{#1}}
\newcommand{\CommentTok}[1]{\textcolor[rgb]{0.56,0.35,0.01}{\textit{#1}}}
\newcommand{\CommentVarTok}[1]{\textcolor[rgb]{0.56,0.35,0.01}{\textbf{\textit{#1}}}}
\newcommand{\ConstantTok}[1]{\textcolor[rgb]{0.00,0.00,0.00}{#1}}
\newcommand{\ControlFlowTok}[1]{\textcolor[rgb]{0.13,0.29,0.53}{\textbf{#1}}}
\newcommand{\DataTypeTok}[1]{\textcolor[rgb]{0.13,0.29,0.53}{#1}}
\newcommand{\DecValTok}[1]{\textcolor[rgb]{0.00,0.00,0.81}{#1}}
\newcommand{\DocumentationTok}[1]{\textcolor[rgb]{0.56,0.35,0.01}{\textbf{\textit{#1}}}}
\newcommand{\ErrorTok}[1]{\textcolor[rgb]{0.64,0.00,0.00}{\textbf{#1}}}
\newcommand{\ExtensionTok}[1]{#1}
\newcommand{\FloatTok}[1]{\textcolor[rgb]{0.00,0.00,0.81}{#1}}
\newcommand{\FunctionTok}[1]{\textcolor[rgb]{0.00,0.00,0.00}{#1}}
\newcommand{\ImportTok}[1]{#1}
\newcommand{\InformationTok}[1]{\textcolor[rgb]{0.56,0.35,0.01}{\textbf{\textit{#1}}}}
\newcommand{\KeywordTok}[1]{\textcolor[rgb]{0.13,0.29,0.53}{\textbf{#1}}}
\newcommand{\NormalTok}[1]{#1}
\newcommand{\OperatorTok}[1]{\textcolor[rgb]{0.81,0.36,0.00}{\textbf{#1}}}
\newcommand{\OtherTok}[1]{\textcolor[rgb]{0.56,0.35,0.01}{#1}}
\newcommand{\PreprocessorTok}[1]{\textcolor[rgb]{0.56,0.35,0.01}{\textit{#1}}}
\newcommand{\RegionMarkerTok}[1]{#1}
\newcommand{\SpecialCharTok}[1]{\textcolor[rgb]{0.00,0.00,0.00}{#1}}
\newcommand{\SpecialStringTok}[1]{\textcolor[rgb]{0.31,0.60,0.02}{#1}}
\newcommand{\StringTok}[1]{\textcolor[rgb]{0.31,0.60,0.02}{#1}}
\newcommand{\VariableTok}[1]{\textcolor[rgb]{0.00,0.00,0.00}{#1}}
\newcommand{\VerbatimStringTok}[1]{\textcolor[rgb]{0.31,0.60,0.02}{#1}}
\newcommand{\WarningTok}[1]{\textcolor[rgb]{0.56,0.35,0.01}{\textbf{\textit{#1}}}}
\usepackage{graphicx}
\makeatletter
\def\maxwidth{\ifdim\Gin@nat@width>\linewidth\linewidth\else\Gin@nat@width\fi}
\def\maxheight{\ifdim\Gin@nat@height>\textheight\textheight\else\Gin@nat@height\fi}
\makeatother
% Scale images if necessary, so that they will not overflow the page
% margins by default, and it is still possible to overwrite the defaults
% using explicit options in \includegraphics[width, height, ...]{}
\setkeys{Gin}{width=\maxwidth,height=\maxheight,keepaspectratio}
% Set default figure placement to htbp
\makeatletter
\def\fps@figure{htbp}
\makeatother
\setlength{\emergencystretch}{3em} % prevent overfull lines
\providecommand{\tightlist}{%
  \setlength{\itemsep}{0pt}\setlength{\parskip}{0pt}}
\setcounter{secnumdepth}{-\maxdimen} % remove section numbering
\ifLuaTeX
  \usepackage{selnolig}  % disable illegal ligatures
\fi

\title{Exercice\_R}
\author{Dan}
\date{2022-03-22}

\begin{document}
\maketitle

\hypertarget{r-markdown}{%
\subsection{R Markdown}\label{r-markdown}}

This is an R Markdown document. Markdown is a simple formatting syntax
for authoring HTML, PDF, and MS Word documents. For more details on
using R Markdown see \url{http://rmarkdown.rstudio.com}.

When you click the \textbf{Knit} button a document will be generated
that includes both content as well as the output of any embedded R code
chunks within the document. You can embed an R code chunk like this:

\begin{Shaded}
\begin{Highlighting}[]
\FunctionTok{summary}\NormalTok{(cars)}
\end{Highlighting}
\end{Shaded}

\begin{verbatim}
##      speed           dist       
##  Min.   : 4.0   Min.   :  2.00  
##  1st Qu.:12.0   1st Qu.: 26.00  
##  Median :15.0   Median : 36.00  
##  Mean   :15.4   Mean   : 42.98  
##  3rd Qu.:19.0   3rd Qu.: 56.00  
##  Max.   :25.0   Max.   :120.00
\end{verbatim}

\hypertarget{including-plots}{%
\subsection{Including Plots}\label{including-plots}}

You can also embed plots, for example:

\includegraphics{Exo_R_PDF_files/figure-latex/pressure-1.pdf}

Note that the \texttt{echo\ =\ FALSE} parameter was added to the code
chunk to prevent printing of the R code that generated the plot.

\begin{verbatim}
    Exercice 1
\end{verbatim}

1- Tableau croisé

\begin{Shaded}
\begin{Highlighting}[]
\FunctionTok{library}\NormalTok{(questionr)}
\end{Highlighting}
\end{Shaded}

\begin{verbatim}
## Warning: package 'questionr' was built under R version 4.1.3
\end{verbatim}

\begin{Shaded}
\begin{Highlighting}[]
\FunctionTok{data}\NormalTok{(}\StringTok{"hdv2003"}\NormalTok{)}
\NormalTok{df }\OtherTok{\textless{}{-}}\NormalTok{ hdv2003}

\CommentTok{\# Tableau}
\FunctionTok{table}\NormalTok{(df}\SpecialCharTok{$}\NormalTok{qualif, df}\SpecialCharTok{$}\NormalTok{clso)}
\end{Highlighting}
\end{Shaded}

\begin{verbatim}
##                           
##                            Oui Non Ne sait pas
##   Ouvrier specialise        76 120           7
##   Ouvrier qualifie         149 137           6
##   Technicien                40  46           0
##   Profession intermediaire  81  78           1
##   Cadre                    148 111           1
##   Employe                  277 311           6
##   Autre                     19  39           0
\end{verbatim}

2- variable dépendante

3- pourcentages ligne ou colonne

\begin{Shaded}
\begin{Highlighting}[]
\NormalTok{tab }\OtherTok{\textless{}{-}} \FunctionTok{table}\NormalTok{(df}\SpecialCharTok{$}\NormalTok{qualif, df}\SpecialCharTok{$}\NormalTok{clso)}
\CommentTok{\# Ligne}
\FunctionTok{lprop}\NormalTok{(tab)}
\end{Highlighting}
\end{Shaded}

\begin{verbatim}
##                           
##                            Oui   Non   Ne sait pas Total
##   Ouvrier specialise        37.4  59.1   3.4       100.0
##   Ouvrier qualifie          51.0  46.9   2.1       100.0
##   Technicien                46.5  53.5   0.0       100.0
##   Profession intermediaire  50.6  48.8   0.6       100.0
##   Cadre                     56.9  42.7   0.4       100.0
##   Employe                   46.6  52.4   1.0       100.0
##   Autre                     32.8  67.2   0.0       100.0
##   All                       47.8  50.9   1.3       100.0
\end{verbatim}

\begin{Shaded}
\begin{Highlighting}[]
\CommentTok{\# Colonne}
\FunctionTok{cprop}\NormalTok{(tab)}
\end{Highlighting}
\end{Shaded}

\begin{verbatim}
##                           
##                            Oui   Non   Ne sait pas All  
##   Ouvrier specialise         9.6  14.3  33.3        12.3
##   Ouvrier qualifie          18.9  16.3  28.6        17.7
##   Technicien                 5.1   5.5   0.0         5.2
##   Profession intermediaire  10.3   9.3   4.8         9.7
##   Cadre                     18.7  13.2   4.8        15.7
##   Employe                   35.1  36.9  28.6        35.9
##   Autre                      2.4   4.6   0.0         3.5
##   Total                    100.0 100.0 100.0       100.0
\end{verbatim}

Interpretation: On peut remarquer qu'en général dans les classes
sociales, les gens ne croient pas à l'existence de classes sociales.
Cependant parmi ceux qui pensent que ces classes existent et même ceux
qui n'y croient pas, les Employés sont plus représentés.

4- Test de ff2

\begin{Shaded}
\begin{Highlighting}[]
\FunctionTok{chisq.test}\NormalTok{(tab)}
\end{Highlighting}
\end{Shaded}

\begin{verbatim}
## Warning in chisq.test(tab): Chi-squared approximation may be incorrect
\end{verbatim}

\begin{verbatim}
## 
##  Pearson's Chi-squared test
## 
## data:  tab
## X-squared = 36.788, df = 12, p-value = 0.0002418
\end{verbatim}

La p-value étant inferieur à 5\%, on peut donc rejetter l'hypothèse
d'indépendance

5- Representation et coloration suivant les Residus

\begin{Shaded}
\begin{Highlighting}[]
\FunctionTok{mosaicplot}\NormalTok{(tab, }\AttributeTok{las =} \DecValTok{2}\NormalTok{, }\AttributeTok{shade =} \ConstantTok{TRUE}\NormalTok{, }\AttributeTok{main =} \StringTok{"Catégorie\&Existence des classes"}\NormalTok{)}
\end{Highlighting}
\end{Shaded}

\includegraphics{Exo_R_PDF_files/figure-latex/unnamed-chunk-5-1.pdf}

\begin{verbatim}
      Exercice 2
      
\end{verbatim}

1- boxplot

\begin{Shaded}
\begin{Highlighting}[]
\FunctionTok{boxplot}\NormalTok{(df}\SpecialCharTok{$}\NormalTok{heures.tv }\SpecialCharTok{\textasciitilde{}}\NormalTok{ df}\SpecialCharTok{$}\NormalTok{occup, }\AttributeTok{xlab =} \StringTok{"Occupation"}\NormalTok{,}
  \AttributeTok{ylab =} \StringTok{"Heures passées devant la TV"}\NormalTok{, }\AttributeTok{main =} \StringTok{"Occupation\&Heures\_TV"}\NormalTok{)}
\end{Highlighting}
\end{Shaded}

\includegraphics{Exo_R_PDF_files/figure-latex/unnamed-chunk-6-1.pdf}

2- la durée moyenne devant la télévision en fonction du statut
d'occupation

\begin{Shaded}
\begin{Highlighting}[]
\FunctionTok{library}\NormalTok{(dplyr)}
\end{Highlighting}
\end{Shaded}

\begin{verbatim}
## Warning: package 'dplyr' was built under R version 4.1.3
\end{verbatim}

\begin{verbatim}
## 
## Attaching package: 'dplyr'
\end{verbatim}

\begin{verbatim}
## The following objects are masked from 'package:stats':
## 
##     filter, lag
\end{verbatim}

\begin{verbatim}
## The following objects are masked from 'package:base':
## 
##     intersect, setdiff, setequal, union
\end{verbatim}

\begin{Shaded}
\begin{Highlighting}[]
\FunctionTok{tapply}\NormalTok{(df}\SpecialCharTok{$}\NormalTok{heures.tv,df}\SpecialCharTok{$}\NormalTok{occup, mean)}
\end{Highlighting}
\end{Shaded}

\begin{verbatim}
## Exerce une profession               Chomeur       Etudiant, eleve 
##                    NA                    NA              1.329787 
##              Retraite   Retire des affaires              Au foyer 
##              2.850255              2.624675              2.822222 
##         Autre inactif 
##              3.265060
\end{verbatim}

\begin{verbatim}
      Exercice 3
\end{verbatim}

1- Nuage de point

\begin{Shaded}
\begin{Highlighting}[]
\CommentTok{\# Charger la base}
\FunctionTok{data}\NormalTok{(}\StringTok{"rp2018"}\NormalTok{)}
\NormalTok{rp }\OtherTok{\textless{}{-}}\NormalTok{ rp2018}

\CommentTok{\# Nuage de points}
\FunctionTok{plot}\NormalTok{(rp}\SpecialCharTok{$}\NormalTok{dipl\_aucun, rp}\SpecialCharTok{$}\NormalTok{proprio, }\AttributeTok{xlab =} \StringTok{"Aucun Diplôme"}\NormalTok{,}
  \AttributeTok{ylab =} \StringTok{"Propriétaire"}\NormalTok{, }\AttributeTok{main =} \StringTok{"Proprietaire\&Personne\_sans\_Diplome"}\NormalTok{)}
\end{Highlighting}
\end{Shaded}

\includegraphics{Exo_R_PDF_files/figure-latex/unnamed-chunk-8-1.pdf}

2- Coeficient de correlation linéaire

\begin{Shaded}
\begin{Highlighting}[]
\FunctionTok{cor}\NormalTok{(rp}\SpecialCharTok{$}\NormalTok{dipl\_aucun, rp}\SpecialCharTok{$}\NormalTok{proprio)}
\end{Highlighting}
\end{Shaded}

\begin{verbatim}
## [1] -0.4064073
\end{verbatim}

\end{document}
